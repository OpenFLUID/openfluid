\chapter{File formats}


An \OFEname \ input dataset includes different informations, shared into many
files:
\begin{itemize}
  \item the spatial domain definition
  \item the flux model definition
  \item the spatial domain input data 
  \item the discrete events
  \item the run configuration
  \item the outputs configuration
\end{itemize}

All these files must be placed into any directory that can be reached by the
engine. The default searched directory is a directory named
\texttt{.openfluid/engine/OPENFLUID.IN} and located into the user home
directory (the user home directory may vary, depending on the used operating
system). If you prefer to place your dataset in another directory, you can
specify it using command line options passed to the engine.


\section{Spatial domain definition (*.ddef.xml)}

The spatial domain is defined by a set of spatial units that are connected each
others. These spatial units are defined by a numerical identifier (ID) and a
class. They also include information about the preocessing order of the unit in
the class. Each unit can be connected to zero or many other units from the
same or a different unit class.\\
\noindent This information is defined through XML files that must end with the
suffix \texttt{.ddef.xml}. All the files in the dataset named using this suffix will be read and considered as spatial domain definition files, and must be
structured follow these rules:
\begin{itemize}
  \item These files are XML files
  \item The root tag must be \texttt{<openfluid>}
  \item Inside the \texttt{<openfluid>} tag, there must be a \texttt{<domain>} tag
  \item Inside the \texttt{<domain>} tag, there must be a set
  of \texttt{<unit>} tags 
  \item Each \texttt{<unit>} tag must bring an \texttt{<ID>} attribute giving
  the identifier of the unit, a \texttt{<class>} attribute giving the class of
  the unit, a \texttt{<pcsorder>} attribute giving the process order in the
  class of the unit \texttt{<class>}
  \item Each \texttt{<unit>} tag may include zero or many \texttt{<to>} tags giving
  the outgoing connections to other units. Each \texttt{<to>} tag must bring an \texttt{<ID>} attribute giving
  the identifier of the connected unit and a \texttt{<class>} attribute giving the class of
  the connected unit  
\end{itemize}

\codeblockfromfile{openfluid-engine_users_en/domain.ddef.xml}

\bigskip

\section{Flux model definition (model.xml)}

\codeblockfromfile{openfluid-engine_users_en/model.xml}

\bigskip

\section{Spatial domain input data (*.ddata.xml)}

\codeblockfromfile{openfluid-engine_users_en/domain.ddata.xml}

\bigskip

\section{Discrete events (*.events.xml)}

\codeblockfromfile{openfluid-engine_users_en/domain.events.xml}

\bigskip

\section{Run configuration(run.xml)}

\codeblockfromfile{openfluid-engine_users_en/run.xml}

\bigskip

\section{Outputs configuration(output.xml)}

\codeblockfromfile{openfluid-engine_users_en/output.xml}

