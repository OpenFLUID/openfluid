\chapter{Usage}

The \OFEname \ application is available on Linux, Windows and MacOSX platforms.
We encourage you to use \OFEname \ program on Linux platform as it is the development and usually used platform. 

\section{Build the input dataset}

Before running the simulation, the input dataset must be built.
An \OFEname \ input dataset includes different informations, shared into many files:
\begin{itemize}
  \item the spatial domain definition
  \item the flux model definition
  \item the spatial domain input data 
  \item the discrete events
  \item the run configuration
  \item the outputs configuration
\end{itemize}

\noindent All these files must be placed into any directory that can be reached by the
engine. The default searched directory is a directory named
\texttt{.openfluid/engine/OPENFLUID.IN} and located into the user home
directory (the user home directory may vary, depending on the used operating
system). If you prefer to place your dataset in another directory, you can
specify it using command line options passed to the engine.\\

\noindent In order to build these files, we encouraged you to use a good text editor, or better, an XML editor.
You can also use custom scripts or macros in specialized sotware, such as spreadsheets or Geographic Information Systems (GIS), to generate automatically the input dataset.

\bigskip

\section{Run the simulation}

\textbf{TODO}

\bigskip

\section{Explore the results}

\textbf{TODO}