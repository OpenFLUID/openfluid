\usepackage{xcolor}
\usepackage{shadow}
\usepackage{ucs}
\usepackage[utf8x]{inputenc}
\usepackage[T1]{fontenc}
\usepackage{graphicx}
\usepackage{array}
\usepackage{verbatim}
\usepackage{tabularx}
\usepackage{ltxtable}
\usepackage{html}
\usepackage{tikz}
\usepackage{cmbright}
\usepackage{listings}


\definecolor{Myorange}{cmyk}{0,0.42,1,0}
\definecolor{Myblue}{rgb}{0.2,0.2,1}
\definecolor{Mygrey}{gray}{0.85}


\definecolor{Myred}{rgb}{1,0.85,0.85}
\definecolor{Myblue}{rgb}{0.85,0.90,1}
\definecolor{Mygreen}{rgb}{0.4,1,0.4}


\definecolor{OFBlue}{HTML}{47617B}
\definecolor{blockblue}{rgb}{0.85,0.90,1}
\definecolor{blockred}{rgb}{1,0.85,0.85}
\definecolor{blockgrey}{gray}{0.85}



% whitecdp (formerly schulzrinne.sty) --provide for blank pages
% between chapters
% This redefinition of the \cleardoublepage command provides
% for a special pagestyle for the "extra" pages which are generated
% to ensure that the chapter opener is on a recto page.
% The pagestyle is "chapterverso"; for many publishers, this should be
% identical to "empty", so that's the default.
% \def\cleardoublepage{\clearpage
%  \if@twoside
%   \ifodd\c@page\else
%    \null\thispagestyle{chapterverso}\newpage
%    \if@twocolumn\null\newpage\fi
%    \fi
%   \fi
%  }%
% \def\ps@chapterverso{\ps@empty}%
\makeatletter
\def\cleardoublepage{\clearpage\if@twoside \ifodd\c@page\else
  \hbox{}
  \thispagestyle{empty}
  \newpage
  \if@twocolumn\hbox{}\newpage\fi\fi\fi}
\makeatother


% macro pour un exemple

% \newcommand{\exempleblock}[1]{
% \begin{latexonly}
% 	\begin{minipage}[t]{0.1\linewidth}
% 		\vspace{-0.5em}
% 	        \includegraphics[scale=0.17]{template/sigle_ex_code.png}
% 	\end{minipage}
% 	\colorbox{Mygrey}{
% 		\begin{minipage}[t]{0.83\linewidth}
% 			\vspace{0.2em}
% 			{{#1}}
% 		\end{minipage}
% 	}
% \end{latexonly}
% }


\newcommand{\coverpage}[3]{
\begin{titlepage}
  \begin{tikzpicture}[remember picture, overlay]
    \draw [fill=OFBlue] (-4,4) rectangle (-0.5,-29.7);
    \draw (7.8,-2.0) node {\includegraphics[width=120mm]{common/graphics/openfluid.png}};
    \draw (16, -15) node [anchor=east] {\huge{\textbf{#1}}};
    \draw (16, -16) node [anchor=east] {\Large{\textbf{#2}}};
    \draw (1,-24) node [anchor=west] {\includegraphics[width=50mm]{common/graphics/LISAH.png}};
    \draw (16,-24) node [anchor=east,text centered,text width=4cm]{\Large{\textit{#3}}};
 \end{tikzpicture}
\end{titlepage}

}




% \newcommand*\chapterlabel{}
% 
% \titleformat{\chapter}
% {\gdef\chapterlabel{}
%  \normalfont\sffamily\Huge\bfseries\scshape}
% {\gdef\chapterlabel{\thechapter\ }}{0pt}
% {\begin{tikzpicture}[remember picture,overlay]
%     \node[yshift=-3cm] at (current page.north west)
%       {\begin{tikzpicture}[remember picture, overlay]
%         \draw[fill=OFBlue] (0,0) rectangle
%           (\paperwidth,3cm);
%         \node[anchor=east,xshift=.9\paperwidth,rectangle,
%               rounded corners=20pt,inner sep=11pt,
%               fill=blue]
%               {\color{white} test};
%        \end{tikzpicture}
%       };
%    \end{tikzpicture}
% }
% \titlespacing*{\chapter}{0pt}{50pt}{-60pt}


\newcommand{\pfname}{OpenFLUID}


% macro pour du code

\newcommand{\codeblock}[2]{
 \begin{latexonly}
 	\colorbox{blockgrey}{
 		\begin{minipage}[t]{1\linewidth}
 			\vspace{0.2em}
 			\lstset{language=#1}
 			#2
% 			\lstlisting{#2}
 		\end{minipage}
 	}
\end{latexonly}
}


\begin{htmlonly}
\renewcommand{\codeblock}[1]{
\begin{rawhtml}
<BR>
<TABLE WIDTH="100%" CELLPADDING="10"><TR><TD WIDTH="80" ALIGN="center" VALIGN="top">
\end{rawhtml}
\includegraphics[scale=1]{template/notecolor.png}
\begin{rawhtml}
</TD><TD BGCOLOR="#eeeeee" ALIGN="left" VALIGN="top">
\end{rawhtml}
	\texttt{#1}
\begin{rawhtml}
</TD></TR></TABLE><BR>
\end{rawhtml}

}
\end{htmlonly}


\newcommand{\codeblockfromfile}[1]{
	\begin{minipage}[t]{0.1\linewidth}
		\vspace{-0.5em}
%	        \includegraphics[scale=0.17]{template/sigle_ex_code.png}
	        \includegraphics[scale=0.8]{template/notecolor.png}
	\end{minipage}
	\colorbox{Mygrey}{
		\begin{minipage}[t]{0.83\linewidth}
			\vspace{0.2em}
			{\footnotesize\verbatiminput{#1}\normalsize}
		\end{minipage}
	}
}


\begin{htmlonly}
\renewcommand{\codeblockfromfile}[1]{
\begin{rawhtml}
<BR>
<TABLE WIDTH="100%" CELLPADDING="10"><TR><TD WIDTH="80" ALIGN="center" VALIGN="top">
\end{rawhtml}
\includegraphics[scale=1]{template/notecolor.png}
\begin{rawhtml}
</TD><TD BGCOLOR="#eeeeee" ALIGN="left" VALIGN="middle">
\end{rawhtml}
	\verbatiminput{#1}
\begin{rawhtml}
</TD></TR></TABLE><BR>
\end{rawhtml}

}
\end{htmlonly}

% % macro pour paragraphe A noter
% \newcommand{\noteblock}[1]{
% 	\hbox{\raisebox{-0.5em}{\vrule depth 0pt height 0.4pt width 1\linewidth}}
% 	\begin{minipage}[t]{0.1\linewidth}
% 		\vspace{-0.6em}
% 	        \includegraphics[scale=0.18]{template/sigle_a_noter.png}
% 	\end{minipage}
% 	\begin{minipage}[t]{0.86\linewidth}
% 		\vspace{0.em}
% 		{\sffamily{#1}}
% 	\end{minipage}
% 	\hbox{\raisebox{-0.2em}{\vrule depth 0pt height 0.4pt width 1\linewidth}}
% }

\newcommand{\noteblock}[1]{
\begin{latexonly}
%	\begin{minipage}[t]{0.1\linewidth}
%      \Huge{Note}
%		\vspace{-0.5em}
%	        \includegraphics[scale=0.24]{template/sigle_a_noter.png}
%	        \includegraphics[scale=0.8]{template/tipcolor.png}
%	\end{minipage}
	\colorbox{blockblue}{
		\begin{minipage}[t]{1\linewidth}
			\vspace{0.2em}
			\sffamily{#1}
		\end{minipage}
	}
\end{latexonly}
}



\begin{htmlonly}
\renewcommand{\noteblock}[1]{
\begin{rawhtml}
<BR>
<TABLE WIDTH="100%" CELLPADDING="10"><TR><TD WIDTH="80" ALIGN="center" VALIGN="top">
\end{rawhtml}
\includegraphics[scale=1]{template/tipcolor.png}
\begin{rawhtml}
</TD><TD BGCOLOR="#E6F0FF" ALIGN="left" VALIGN="middle">
\end{rawhtml}
	\sffamily{#1}
\begin{rawhtml}
</TD></TR>
</TABLE><BR>
\end{rawhtml}

}
\end{htmlonly}



% macro pour paragraphe attention!
% \newcommand{\warnblock}[1]{
% 	\hbox{\raisebox{-0.5em}{\vrule depth 0pt height 0.4pt width 1\linewidth}}
% 	\begin{minipage}[t]{0.1\linewidth}
% 		\vspace{-0.6em}
% 	        \includegraphics[scale=0.24]{template/sigle_warn.png}
% 	\end{minipage}
% 	\begin{minipage}[t]{0.86\linewidth}
% 		\vspace{0.em}
% 		{\sffamily{#1}}
% 	\end{minipage}
% 	\hbox{\raisebox{-0.2em}{\vrule depth 0pt height 0.4pt width 1\linewidth}}
% }


% macro pour paragraphe attention!
\newcommand{\warnblock}[1]{
\begin{latexonly}
	\colorbox{blockred}{
		\begin{minipage}[t]{1\linewidth}
			\vspace{0.2em}
			{\sffamily{#1}}
		\end{minipage}
	}
\end{latexonly}
}


\begin{htmlonly}
\renewcommand{\warnblock}[1]{
\begin{rawhtml}
<BR>
<TABLE WIDTH="100%" CELLPADDING="10"><TR><TD WIDTH="80" ALIGN="center" VALIGN="top">
\end{rawhtml}
\includegraphics[scale=1]{template/warncolor.png}
\begin{rawhtml}
</TD><TD ALIGN="left" VALIGN="middle" BGCOLOR="#FFE6E6">
\end{rawhtml}
	\sffamily{#1}
\begin{rawhtml}
<BR></TD></TR>
</TABLE><BR>
\end{rawhtml}

}
\end{htmlonly}

