\subsection{Nom de la fonction}
Le nom (fileID) de la fonction de simulation est \texttt{\FileID}.

\subsection{Paramètres de la fonction}
La fonction ``\frenchname'' doit être utilisée et renseignée avec les paramètres suivants :\\


\hspace{-0.5cm}
\begin{tabular}{|llcc|}
 \hline
\it Symbole & \it Nom & \it Valeurs & \it Unité \\
 \hline
$Max Steps$ & \texttt{\ParamA} & $>0$ & $-$ \\
$C$ & \texttt{\ParamB} & $>0$ & $m/s$ \\
$D$ & \texttt{\ParamC} & $>0$ & $m\up2/s$ \\
\hline
\end{tabular} 
\vspace{1em}

La syntaxe correcte de déclaration d'utilisation de la fonction dans le fichier \texttt{model.xml} doit ressembler à l'exemple illustré ci-après :

\begin{verbatim}
<function fileID="water.surf.
          transfer-su.hayami">
    <param name="maxsteps" value="100" />
    <param name="meancel" value="0.05" />
    <param name="meansigma" value="500" />
</function>
\end{verbatim}


\subsection{Propriétés distribuées}
La fonction ``\frenchname'' requiert les propriétés distribuées suivantes :
\vspace{1em}

\hspace{-0.5cm}
\begin{tabular}{|llcc|}
 \hline
\it Symbole & \it Nom & \it Valeurs & \it Unité \\
 \hline
$n$ & \texttt{\PropDisA} & $>0$ & $s/m\up{-1/3}$ \\
$A$ & \texttt{\PropDisB} & $>0$ & $m2$ \\
$\beta$ & \texttt{\PropDisC} & $>0$ & $m/m$ \\
$d$ & \texttt{\PropDisD} & $>0$ & $m$ \\
\hline
\end{tabular} 
\vspace{1em}


\subsection{Variables}
Les variables produite, utilisée et requise par cette fonction sont listées dans le tableau ci-après.
\vspace{1em}

\hspace{-0.5cm}
\begin{tabular}{|lll|}
 \hline
\it Symbole & \it Nom & \it Unité \\
 \hline
$R$ & \texttt{\VarRequiredA} & $m$ \\
$Q_e$ & \texttt{\VarUsedA} & $m\up{3}/s$ \\
$Q_{SU}$ & \texttt{\VarProdA} & $m\up{3}/s$ \\
\hline
\end{tabular} 
\vspace{1em}