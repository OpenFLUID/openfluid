\section{Description fonctionnelle}
\subsection{Nom de la fonction}
Le nom (fileID) de la fonction de simulation est \texttt{\FileID}.

\subsection{Paramètres de la fonction}
La fonction ``\frenchname'' doit être utilisée et renseignée avec le paramètre suivant :\\


\hspace{-0.5cm}
\begin{tabular}{|llcc|}
 \hline
\it Symbole & \it Nom & \it Valeurs & \it Unité \\
 \hline
$P_{threshold}$ & \texttt{\ParamA} & $\geq 0$ & $m$ \\
\hline
\end{tabular} 
\vspace{1em}

La syntaxe correcte de déclaration d'utilisation de la fonction dans le fichier \texttt{model.xml} doit ressembler à l'exemple illustré ci-après :

\begin{verbatim}
<function fileID="water.atm-surf.
          rain-su.files">
    <param name="threshold"
           value="0.0001" />
</function>
\end{verbatim}


\subsection{Variables}
La variable produite par cette fonction est indiquée dans le tableau ci-après :
\vspace{1em}

\hspace{-0.5cm}
\begin{tabular}{|lll|}
 \hline
\it Symbole & \it Nom & \it Unité \\
 \hline
$P$ & \texttt{\VarProdA} & $m$ \\
\hline
\end{tabular} 
\vspace{1em}


\subsection{Fichiers}
La fonction ``\frenchname'' nécessite plusieurs fichiers :\\
\vspace{1em}

\hspace{-0.5cm}
\begin{tabular}{|l|}
 \hline
\it Nom du fichier \\
 \hline
rainsources.xml \\
SUraindistri.dat \\
\hline
\end{tabular} 
\vspace{1em}
