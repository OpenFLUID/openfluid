\subsection{Nom de la fonction}
Le nom (fileID) de la fonction de simulation est \texttt{\FileID}.

\subsection{Paramètres de la fonction}
La ``\frenchname'' doit être utilisée et renseignée avec le paramètre suivant :\\


\hspace{-0.5cm}
\begin{tabular}{|llcc|}
 \hline
\it Symbole & \it Nom & \it Valeurs & \it Unité \\
 \hline
$Res Step$ & \texttt{\ParamA} & $>0$ & $m$ \\
\hline
\end{tabular} 
\vspace{1em}

La syntaxe correcte de déclaration d'utilisation de la fonction dans le fichier \texttt{model.xml} doit ressembler à l'exemple illustré ci-après :

\begin{verbatim}
<function fileID="water.surf-uz.runoff
          -infiltration.mseytoux">
    <param name="resstep"
           value="0.000005" />
</function>
\end{verbatim}


\subsection{Propriétés distribuées}
La ``\frenchname'' requiert les propriétés distribuées suivantes :
\vspace{1em}

\hspace{-0.5cm}
\begin{tabular}{|llcc|}
 \hline
\it Symbole & \it Nom & \it Valeurs & \it Unité \\
 \hline
$K_s$ & \texttt{\PropDisA} & $\geq 0$ & $m/s$ \\
$\theta_r $ & \texttt{\PropDisB} & $0 \le \theta_r \le 1$ & $m\up{3}/m\up{3}$ \\
$\theta_s$ & \texttt{\PropDisC} & $0 \le \theta_s \le 1$ & $m\up{3}/m\up{3}$ \\
$\beta_{MS}$ & \texttt{\PropDisD} & $>0$ & $-$ \\
$H_c$ & \texttt{\PropDisE} & $\geq 0$ & $m$ \\
$A_{SU}$ & \texttt{\PropDisF} & $>0$ & $m\up2$ \\
\hline
\end{tabular} 
\vspace{1em}

La condition $\theta_r \le \theta_s$ devra être vérifiée.\\


\subsection{Conditions initiales}
Pour chaque SU, une condition initiale est nécessaire et doit être présente dans le fichier \texttt{SUini.ddata.xml}. Cette condition initiale est la suivante :
\vspace{1em}

\hspace{-0.5cm}
\begin{tabular}{|llcc|}
 \hline
\it Symbole & \it Nom & \it Valeurs & \it Unité \\
 \hline
$\theta_i$ & \texttt{\InitA} & $0 \le \theta_i \le 1$ & $m\up{3}/m\up{3}$ \\
\hline
\end{tabular} 
\vspace{1em}

La condition $\theta_r \le \theta_i \le \theta_s$ devra être vérifiée.\\


\subsection{Variables}
Les variables produites, utilisées et requises par cette fonction sont listées dans le tableau ci-après :
\vspace{1em}

\hspace{-0.5cm}
\begin{tabular}{|lll|}
 \hline
\it Symbole & \it Nom & \it Unité \\
 \hline
$R$ & \texttt{\VarProdA} & $m$ \\
$I$ & \texttt{\VarProdB} & $m$ \\
$Q_{SU}$ & \texttt{\VarUsedA} & $m\up{3}/s$ \\
$K_s$ & \texttt{\VarUsedB} & $m/s$ \\
$P$ & \texttt{\VarRequirA} & $m$ \\
\hline
\end{tabular} 
\vspace{1em}